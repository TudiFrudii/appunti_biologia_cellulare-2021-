\Large\textbf{Sigle}\\
\small \\
Di seguito elencate le sigle utilizzate nel testo.

\small
\begin{itemize}
    \item MT: microtubuli
    \item MF: microfilamenti
    \item FI: filamenti intermedi
    \item RE: reticolo endoplasmatico
    \item pb, b: paia di basi, basi
    \item AA: amminoacido/amminoacidico
    \item $\alpha E$: alpha Elica
    \item $F\beta$: Foglietto beta
    \item ORI: origine di replicazione
    \item DNA-P: DNA polimerasi
    \item RNA-P: RNA polimerasi
    \item mRNA: RNA messaggero
    \item rRNA: RNA ribosomiale
    \item tRNA: RNA transfer
    \item miRNA: micro RNA
    \item $\alpha$T: $\alpha$ tubulina
    \item $\beta$T: $\beta$ tubulina
    \item $\gamma$T: $\gamma$ tubulina
    \item CHI: chinesina (generica)
    \item CHI2: chinesina 2
    \item CHI5: chinesina 5
    \item CHI13: chinesina 13
    \item DIN: dineina (generica)
    \item GEF: Guanine nucleotide exchange factor
    \item GAP: GTPase activating protein
    \item CPC: Chromosomal Passenger Complex
    \item MI: miosina (generica)
    \item MI2: miosina 2
    \item MI5: miosina 5
    \item wt: wild type
    \item UB: ubiquitina
    \item UPR: unfolded protein response
    \item TS: tagreting sequence
    \item STA: stop transfer anchor
    \item SA: anchor signal
    \item NPC: nuclear pore complex
    \item NAP: nucleoporine
    \item NLS: nuclear localization signal
    \item NES: nuclear export sequence
    \item RAN: RAs-related Nuclear protein
    \item DNA-M: DNA mitocondriale
    \item DNA-C: DNA cellulare
    \item MTS: sequenza target mitocondriale
\end{itemize}

