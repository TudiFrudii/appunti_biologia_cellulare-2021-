\Large\textbf{Capitolo 1: \\Molecole, cellule e organismi modello}


\section{Molecole della Vita}
\small
Ogni sistema biologico che si è sviluppato fino ad ora, segue gli stessi principi fisici e chimici e ognuno di essi è strutto della pressione selettiva. Possiamo dire che ogni organismo sviluppatosi è dunque derivante da un singolo genitore: \textit{LUCA (last universal common ancestor)}, un organismo unicellulare.

//parte comune

\subsection{Metabolismo}
\small
Gli organismi riescono a vivere e svolgere le loro funzioni vitali grazie al metabolismo, il quale viene diviso in anabolismo e catabolismo:
\begin{itemize}
    \item catabolismo: si processano molecole per scomporle e trarne energia
    \item anabolismo: si processano molecole per comporre strutture più grandi, utilizzando energia
\end{itemize}

\subsection{Dogma centrale della biologia}
\small
Il patrimonio genetico viene custodito all'interno della cellula grazie agli acidi nucleici: il DNA è incluso in ogni cellula dell'organismo vivente in copia uguale. \\
Lo stesso DNA è la base per la produzione di ogni proteina che fa parte dell'organismo: il Dogma Centrale della Biologia infatti esplica l'importanza del DNA per produrre mRNA che in ultimo verrà tradotto in proteine dai ribosomi.


\section{Cellula procariote}
La struttura dei procarioti risulta poco articolata se confrontata con la struttura eucariotica. I procarioti possiedono una parete cellulare (con strutture diverse tra Gram+ e Gram-), membrana cellulare (due nel caso dei Gram-), citoplasma e nucleoide. In quest'ultimo è incluso il materiale genetico, generalmente molto meno corposo rispetto a genomi di eucarioti. 
//scrivi il resto, guarda appunti di micro

  \begin{figure}[h]
            \centering
            \includegraphics[width=1\textwidth]{images/schemagenerale.JPG}
            \caption{\small cellula animale e vegetale a confronto}
            \label{fig:mesh1}
        \end{figure}

\section{Cellula eucariote}
La struttura interna di una cellula eucariotica risulta più compartimentata e articolata rispetto a quella procariotica. La struttura stessa può variare nello stesso organismo a seconda per esempio del tessuto del quale fa parte. La cellula eucariote è composta da:
\begin{itemize}
    \item membrana plasmatica: controlla il trasporto delle sostanze dall'interno all'esterno e viceversa, si occupa di signaling e adesione
    \item mitocondri: sono organelli circondati da una doppia membrana: quella esterna ha pori grandi e funge da filtro mentre quella interna è più selettiva, lo spazio compreso nella membrana interna si chiama matrice mitocondriale. Generano ATP per ossidazione di glucosio o acidi grassi. Responsabile per la morte cellulare. In una cellula raramente è presente un unico mitocondrio: sono bensì presenti degli agglometari di organelli che comunicano tra loro. Contengono DNA proprio, in particolare un cromosoma circolare (vedi teoria endosimbiontica).
    \item lisosomi: degradano materiale interno alla cellula e materiale cellulare "usurato" o malfunzionante (in quest'ultimo caso si parla di autofagia)
    \item membrana nucleare: doppia membrana che circonda il nucleo, la membrana esterna è in continuità con il RE
    \item nucleo: è un organello dove avviene la sintesi di DNA e mRNA
    \item cromatina: è composta da DNA e proteine
    \item REL: il reticolo endoplasmatico ruvido sintetizza e processa le proteine
    \item RER: il reticolo endoplasmatico liscio contiene enzimi che sintetizzano i lipidie de-tossificano alcune molecole idrofobiche
    \item apparato del Golgi: processa e smista le proteine secrete dal RE e lipidi, è strutturato a cisterne in continuità con il RE
    \item vescicole secretorie: sono vescicole che contengono sostanze che devono essere secrete all'esterno o che viceversa (si fondono con la membrana per permettere il passaggio). Queste operazioni richiedono energia
    \item perossisomi: contengono enzimi che rompono gli acidi grassi
    \item fibre citoscheletriche: consistono nella rete di collegamento intracellulare, sono coinvolte nel movimento della cellula e nel mantenimento della sua struttura. 
    Si dividono in microtubuli (MT), filamenti intermedi (FI) e microfilamenti (MF)
    \item microvilli: aumentano la superficie di assorbimento di nutrienti \\
    \item parete cellulare: nelle cellule vegetali mantiene la forma cellulare e fornisce protezione
    \item vacuolo: nelle cellule vegetali mantiene acqua e nutrienti, degrada macromolecole e serve all'elongazione della cellula durante la crescita
    \item cloroplasti: nelle cellule vegetali compiono il processo di fotosintesi, sono circondati da una doppia membrana, hanno un DNA proprio (vedi teoria endosimbiontica).
    \item plasmodesmata: (?) nelle cellule vegetali, sono giunzioni che connettono il citoplasma di una cellula con quello di altre cellule adiacenti
    \item citoplasma: per esclusione si definisce citoplasma tutto ciò interno alla cellula che non si identifica in un organello.
    
\end{itemize}

       

\section{Organismi modello}

\subsection{Saccharomyces cerevisiae}
Il lievito (\textit{Saccharomyces cerevisiae}) è un organismo eucariote composto di una sola cellula. Proprio per questo motivo è stato ampiamente utilizzato come organismo modello e studiato a fondo. 
Il suo ciclo di vita consiste in una fase aploide e una fase diploide. Un altro motivo per il suo ampio utilizzo è proprio la velocità di duplicazione dovuta alla fase aploide che velocizza l'analisi in ricerca genetica. Risulta inoltre facile da manipolare e poco costoso economicamente. 
Il suo genoma è composto da circa 12.6 milioni di paia di basi (pb). Per dare un confronto, Escherichia coli ne possiede 4.6 milioni e il genoma umano oltre 3.2 miliardi. Possiede sequenze introniche e il meccanismo di espressione dei geni è simile a quella dell'uomo.

\subsection{Chlamydomonas reinhardtii}
L'alga unicellulare \textit{chlamydomonas reinhardtii} è "l'analogo vegetale" per il lievito visto nel paragrafo precedente. Contiene cloroplasti ed effettua la fotosintesi. Anche in questo caso, è un organismo facile da manipolare

\subsection{Caenorhabditis elegans}
Il nematode \textit{Caenorhabditis elegans} è un organismo pluricellulare conosiuto anche come \textit{Roundworm}. Ha fornito la base per molti studi, in particolare per quelli che riguardano la morte cellulare programmata. Da adulto, ha un numero costante di cellule pari a 959

\pagebreak